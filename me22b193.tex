
\documentclass{article}
\begin{document}
\section{ME22B193}
\subsection{\underline{Equation of momentum}}
\label{eqn}
\begin{equation}
    p=mv
\end{equation}
The equation represents the momentum of an object and is widely used in physics to describe the behavior of objects in motion. Momentum is a measure of an object's motion and is defined as the product of its mass and velocity. The equation states that the momentum of an object is equal to the product of its \textbf{ mass} (m) and \textbf{ velocity} (v). This relationship between momentum and mass and velocity is one of the fundamental principles of mechanics. In physics, momentum is a vector quantity. When an object is in motion, it has kinetic energy, which is proportional to its mass and velocity. The greater the mass and velocity of an object, the greater its momentum and kinetic energy.The equation is useful for calculating the behavior of objects in motion, such as in collisions and explosions.\\ 
NAME: SHIFA GANI BASHEER\\
GITHUB USER ID: shifa-gani\\
ROLL NO: ME22B193

\footnote{Reference:  Jearl Walker, Robert Resnick, David Halliday (2017). Fundamentals of Physics.}
\end{document}

