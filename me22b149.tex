\section{me22b149}
Name: Madabathula Revanth\\
GitHub user-ID : ME22b149Revanth
\subsection*{Description\footnote{Taken from Quora}}
A Fourier transform (FT) is a mathematical transform that decomposes functions into frequency components, which are represented by the output of the transform as a function of frequency. Most commonly functions of time or space are transformed, which will output a function depending on the temporal frequency or spatial frequency respectively. That process is also called analysis.\\
\\
The term Fourier transform refers to both the frequency domain representation and the mathematical operation that associates the frequency domain representation to a function of space or time.
\subsection*{The analysis formula\footnote{Taken from Wikipedia}}
The Fourier transform is an extension of the Fourier series, which in its most general form introduces the use of complex exponential functions. For example, for a function \textit{f(x)}, the amplitude and phase of a frequency component at frequency \textit{n/P} , \textit{n} $\in$ $\mathbb{Z}$ is given by this complex number: \\

\begin{equation}
\label{a1}
    c_n = \frac{1}{P}\int_P f(x) e^{-i2\pi \frac{n}{P} x }dx \
\end{equation}

The extension provides a frequency continuum of components ,  ( $\xi$ $\in$ $\mathbb{R}$ ), using an infinite integral of integration:\\
\begin{equation}
\label{a2}
    \hat{f} (\xi)= \int_{-\infty}^\infty f(x) e^{-i2 \pi \xi x} dx
\end{equation}
Here, the transform of the function $\textit{f(x)}$ at frequency $\xi$ is denoted by the complex number $\hat{f}$($\xi$), which is just one of several common conventions. Evaluating \ref{a2} for all values of $\xi$ produces the frequency-domain function. When the independent variable ($\textit x$) represents time (often denoted by $\textit{t}$), the transform variable ($\xi$ ) represents frequency (often denoted by $\textit{f}$).

