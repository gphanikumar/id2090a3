

\section{Introduction}
I am Ananya S Krishna, github id; ananya-krishna-13, the author of this file. Here I will brief about Bhor's quantization of angular momentum.
Bohr’s atomic model laid down various postulates for the arrangement of electrons in different orbits around the nucleus. According to Bohr’s atomic model, the angular momentum of electrons orbiting around the nucleus is quantized. He further added that electrons move only in those orbits where the angular momentum of an electron is an integral multiple of h/2. This postulate regarding the quantisation of angular momentum of an electron was later explained by Louis de Broglie. According to him, a moving electron in its circular orbit behaves like a particle-wave.

The angular momentum of an electron by Bohr is given by mvr or nh/2π (where v is the velocity, n is the orbit in which the electron is revolving, m is mass of the electron, and r is the radius of the nth orbit).

\begin{equation}
    mvr=nh/2*($\pi$)
\end{equation}

\end{document}
\footnote{https://byjus.com/physics/angular-momentum-of-electron/#:~:text=The%20angular%20momentum%20of%20an%20electron%20by%20Bohr%20is%20given,radius%20of%20the%20nth%20orbit).}
