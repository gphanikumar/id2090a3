\section{ME22B173}

Name: Ojas Wani\\
Roll No.: ME22B173\\
GitHub ID: OjasW


\subsection{Dirac Equation\protect\footnote{${https://en.wikipedia.org/wiki/Dirac_equation}$}}

In particle physics, the Dirac equation is a relativistic wave equation derived by British physicist Paul Dirac in 1928. In its free form, or including electromagnetic interactions, it describes all spin-1⁄2 massive particles, called "Dirac particles", such as electrons and quarks for which parity is a symmetry. It is consistent with both the principles of quantum mechanics and the theory of special relativity, and was the first theory to account fully for special relativity in the context of quantum mechanics. \\
In its modern formulation for field theory, the Dirac equation is written in terms of a Dirac spinor field $\psi$  taking values in a complex vector space described concretely as ${\displaystyle \mathrm {C} ^{4}}$, defined on flat spacetime (Minkowski space) ${\displaystyle \mathrm {R} ^{1,3}}$. Its expression also contains gamma matrices and a parameter $m > 0$ interpreted as the mass, as well as other physical constants. \newline
\\
\centerline{In terms of a field ${\displaystyle \psi :\mathrm {R} ^{1,3}\rightarrow \mathrm {C} ^{4}}$, the Dirac equation is then -} \\ \centerline{${\displaystyle i\hbar \gamma ^{\mu }\partial _{\mu }\psi (x)-mc\psi (x)=0}$} \newline
\\
The gamma matrices are a set of four 4 $\times$ 4 complex matrices (elements of ${\displaystyle {\text{Mat}}_{4\times 4}(\mathrm {C} )})$ which satisfy the defining anti-commutation relations:

\centerline{${\displaystyle \{\gamma ^{\mu },\gamma ^{\nu }\}=2\eta ^{\mu \nu }I_{4}}$} 
where $\eta ^{\mu \nu }$ is the Minkowski metric element, and the indices $\mu, \nu$ run over 0, 1, 2 and 3. These matrices can be realized explicitly under a choice of representation. Two common choices are the Dirac representation

\centerline{${\displaystyle \gamma ^{0}={\begin{pmatrix}I_{2}&0\\0&-I_{2}\end{pmatrix}},\quad \gamma ^{i}={\begin{pmatrix}0&\sigma ^{i}\\-\sigma ^{i}&0\end{pmatrix}},}$}


where $\sigma ^{i}$ are the Pauli matrices, and the chiral representation: the $\gamma^i$ are the same, but ${\displaystyle \gamma ^{0}={\begin{pmatrix}0&I_{2}\\I_{2}&0\end{pmatrix}}.}$
