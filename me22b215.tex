\section{ME22B215}
\subsection{Schrodinger's Wave Equation }
The \textbf{Schrodinger equation} is a linear partial differential equation that governs the wave function of a quantum-mechanical system.It is a key result in quantum mechanics, and its discovery was a significant landmark in the development of the subject. The equation is named after Erwin Schrodinger, who postulated the equation in 1925, and published it in 1926, forming the basis for the work that resulted in his Nobel Prize in Physics in 1933.
\begin{equation}
    \hat{\textbf{H}} \bigl| \psi\rangle = E\bigl| \psi\rangle
\end{equation}
\begin{equation}
-\frac{\hbar^2}{2m}\nabla^2\psi(x,y,z)+V(x,y,z)\psi(x,y,z)=E\psi(x,y,z)
\end{equation}

The $\psi(x,y,z)$ is a complex function which describes the amplitude of the particle. $\psi^2(x,y,z)$ describes the probability density of finding an electron in a region.

\begin{equation}
(\frac{\partial^2 \psi}{\partial x^2}+\frac{\partial^2 \psi}{\partial y^2}+\frac{\partial^2 \psi}{\partial z^2})+\frac{2m}{\hbar^2}(E-V)\psi(x,y,z)=0
\end{equation}
Solving the differential equation with boundary conditions gives us the structure of orbitals.

The Schrodinger equation is not the only way to study quantum mechanical systems and make predictions. The other formulations of quantum mechanics include matrix mechanics, introduced by Werner Heisenberg, and the path integral formulation, developed chiefly by Richard Feynman.
\\\\
The above is a extract from Wikkipedia \\
link :"https://en.wikipedia.org/wiki/Schr\"{o}dinger\_equation"
\\\\
Name: \textbf{G R YOGESH} \\
Git-hub user-id: \textbf{YogeshRajasekhar}
