\documentclass[12pt,a4paper]{article}
\usepackage{amsmath}
\usepackage{amsfonts}
\usepackage{hyperref}
\usepackage[left=2cm,right=2cm,top=2cm,bottom=2cm]{geometry}
\author{Adithya Rama Varma}
\title{\textbf{me22b012.tex}}
\begin{document}

\maketitle

\section{me22b012 }
\begin{equation}
{\tau}(x)=\int_{0}^{\infty}{\text{exp}(-t)}{\text (t^{\text x-1})dt}
\end{equation}
It is a extension of factorial function to complex numbers.It is defined by a convergent improper integral.
If the real part of the complex number z is strictly positive , then the integral converges absolutely, and is known as the Euler integral of the second kind.
The notation tau is due to Legendre.
for every positive integer n
\begin{equation}
{\tau}(n)=(n-1)!
\end{equation}
There exist the recuirsion relation 
\begin{equation}
{\tau}(z+1)=z{\tau}(z)
\end{equation}
This is derived by integrating in parts
now the gamma function can also be represented in the form 
\begin{equation}
n!=\int_{0}^{1}{\text (- log x)^{\text -n}dx}
\end{equation}
The gamma function can be related to the first Euclerian integral or the beta function as follows
\begin{equation}
B(m,n)=\frac {{\tau}(m) {\tau}(n)}{{\tau}(m+n)}
\end{equation}
\emph{Name-T N Adithya Rama Varma }\\
\emph{GitHub id-thedarkknight-012}\\
\footnote{reference\url{https://www.maa.org/sites/default/files/pdf/upload_library/22/Chauvenet/Davis.pdf}}
\footnote{reference\url{https://www.roma1.infn.it/~bonvini/math/Marco_Bonvini__Gamma_function.pdf}}
\end{document}
