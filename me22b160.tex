\section{ME22B160}
\subsection{A very interesting summation given by Euler}
\begin{equation}
    e^x = \sum_{n=0}^{\infty} \frac{x^n}{n!}
\end{equation}
\subsection{Introduction}
This is a summation which transforms the power function of e into a beautiful summation.

Euler must have been a real genius to discover such a number. The number itself is a really interesting one. For more information, refer to this video of Numberphile. \footnote{\href{https://www.youtube.com/watch?v=AuA2EAgAegE}{Numberphile's video on Euler's Number}}
\subsection{Uses of the exponential function}
The sum has much more use over the number itself. Some hyperbolic functions are the literal transformation derived from this summation itself. Another intersting use of this summation is in the fields of complex numbers and Binomial Theorem. Refer to the wikipedia link for more information on this. \footnote{\href{https://en.wikipedia.org/wiki/Exponential_function}{Wikipedia link}}
\begin{itemize}
    \item Use in BT:- Some complex bino-binomial sums are calculated using power functions and consequently, power sum of euler's number.
    \item Use in Complex Numbers:- Euler's form of complex numbers is somewhat derived from this expression.
\end{itemize}