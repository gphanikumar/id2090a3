\section{ME22B004}
  Torque is a measure of rotational force or the force that causes an object to rotate around an axis. It's like a twisting force that makes an object turn.

  Think of a door that is attached to a hinge. If you push or pull the door, you are applying a force to it. This force will cause the door to rotate around the hinge, which is the axis of rotation. The amount of rotational force you apply to the door is known as torque.

  In physics, torque is represented mathematically as a vector and is calculated as the cross product of the lever arm and the force applied. The lever arm is a vector that extends from the axis of rotation to the point where the force is applied. The direction of the torque vector is perpendicular to the plane formed by the lever arm and the force vector
  \footnote{https://www.physics.uoguelph.ca/torque-and-rotational-motion-tutorial}
  
{$$ \textbf{$\tau$}= \textbf{rf}sin \theta$$}

$\tau$=torque

r=radius

F=force

$\theta$=angle between F and the lever arm


