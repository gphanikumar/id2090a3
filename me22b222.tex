\section{ME22B222}
The first mention of the binomial theorem was in the 4th century BC by a famous Greek mathematician by name of Euclids. The binomial theorem states the principle for expanding the algebraic expression $(x + y)^n$ and expresses it as a sum of the terms involving individual exponents of variables x and y. Each term in a binomial expansion is associated with a numeric value which is called coefficient.
\\
According to the binomial theorem, it is possible to expand any non-negative power of binomial (x + y) into a sum of the form
\begin{equation}
    (x+y)^n={}^{n}C_{0}x^ny^0+{}^{n}C_{1}x^{n-1}y^1+{}^{n}C_{2}x^{n-2}y^2+.......+{}^{n}C_{n-1}x^{1}y^{n-1}+{}^{n}C_{n}x^0y^n
\end{equation}
where, $ n\ge 0 $  is an integer and each ${}^{n}C_{k}$ is a positive integer known as a binomial coefficient and can be calculated using the formula
\begin{equation}
     {}^{n}C_{k}= n!/(k! * (n-k)!)
\end{equation}
The Binomial Theorem has numerous applications in fields such as combinations, probability, and algebra. It provides a convenient way to expand and simplify expressions and is a key tool in solving a wide range of mathematical problems.
\footnote{REFERENCE:-https://www.cuemath.com/algebra/binomial-theorem/}
\\
\\
NAME:-ROSHAN ZAMBARE  \\
GITHUB ID:-Roshan21312
