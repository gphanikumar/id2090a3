


\section*{MALIK RAKESH ME22B150 Github:Rakesh3m}
\section{Doppler Effect}
It is an everyday experience that the When we approach a stationary Source of Sound heard appears to be higher than that of the source. As the Observer recedes away from the Source,the Observed Pitch(frequency) becomes lower than that of Source. This motion-related frequency change is called DOPPLER EFFECT.  


\begin{equation}
    f=f_o(\frac{v-v_o}{v-v_s})
\end{equation}\footnote{REFERENCE: NCERT 11th}

where,
\begin{center}

   f= observed frequency\\\
    fo= actual frequency\\\
    v = velocity of sound waves\\\
    $v_o$ = velocity of observer\\\
    $v_s$= velocity of source
    
\end{center} 






 Case 1: Source Moving Towards the Observer at Rest \\\
  in this case, the observer's velocity is zero,so $v_o$ is equal to zero.substituting this into the doppler equation above,we get the equation of Doppler effect when a source is moving towards an observer at rest as: 
 
 \begin{equation}
     f=f_o(\frac{v}{v-v_s})
 \end{equation}

 Case 2: Source moving away from the Observer at Rest\\\                
  Since the velocity of the observer is zero,we can eliminate
  $v_o$ from the equation.But this time,the source moves away from the observer,so it's velocity is negative to indicate the direction. Hence,the equation now becomes as follows:

\begin{equation}
    f=f_o(\frac{v}{v-(-v_s)})
\end{equation}

Case 3: Observer Moving Towards a Stationary Source\\\
in this case $v_s$ will equal to zero,hence we get the following equation:
\begin{equation}
    f=f_o(\frac{v+v_o}{v})
\end{equation} 

Case 4: Observer Moving away from stationary source\\\
Since the observer is moving away,the velocity of the observer becomes nehgative. So, instead of adding $v_o$, we now subtract,since $v_o$ is negative.
\begin{equation}
    f=f_o(\frac{v-v_o}{v})
\end{equation}
 
 

 

