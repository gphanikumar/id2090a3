\section{ME22B044} 
\begin{center}
  \huge{RELATION BETWEEN RESISTANCE AND TEMPERATURE} \footnote{https://en.wikipedia.org/wiki/Electrical_resistance_and_conductance}
\end{center}
\begin{center}
\bigskip
  \huge{  {R}(T) = {R_0}( 1 + {\alpha}(\Delta{T}) ) }
\bigskip
  \label{main_eqn}

  \medium{  where     
  {\Delta{T} = T - T_0}}
\bigskip
\end{center}

\(\alpha \) is called the temperature coefficient of resistance, \(T_0 \) is a fixed reference temperature (usually room temperature), and \(R_0 \) is the resistance at the temperature \(T_0 \). The parameter \(\alpha \) is an empirical parameter fitted from measurement data. Because the linear approximation is only an approximation,\(\alpha \) is different for different reference temperatures. For this reason it is usual to specify the temperature that \(\alpha \) was measured at with a suffix, such as \(\(\alpha \)_15 \)  and the relationship only holds in a range of temperatures around the reference.The temperature coefficient \(\alpha \) is typically \(3{\times}10^-3 \) \(K^-1 \) to \(6{\times}10^-3 \) \(K^-1 \) for metals near room temperature. It is usually negative for semiconductors and insulators, with highly variable magnitude.

\bigskip
\begin{flushleft}
Name: AKSHAY KUMAR G\\
Github User ID: G-Akshaykumar
\end{flushleft}
