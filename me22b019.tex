

\title{ID assignment 3}
\author{Sarvesh Raosaheb Mane me22b019}
\date{February 2023}

\maketitle

\section{me22b019}
The equation of motion of \emph{damped} and \emph{driven} simple harmonic oscillator is:
\footnote{\textbf{References}}

\[\frac{d^2x}{dt^2} + 2\gamma\frac{dx}{dt} + \omega^2x = \frac{F(t)}{m} \] \\
where $2\gamma= \frac{b}{m}$, $\omega^2= \frac{k}{m}$
\footnote{Introduction to Classical
Mechanics by David Morin} \\

Any \emph{natural oscillator}, left to itself, eventually comes to rest, as the inevitable damping
forces drain its energy. Thus if one wants the oscillations to continue, one must arrange
for some \textbf{extemal "driving" force} to maintain them.\\
\textit{ For example},
\footnote{Classical Mechanics by John R Taylor}
\begin{itemize}
    \item the motion of the
pendulum in a grandfather clock is driven by periodic pushes caused by the clock's
weights

 \item the motion of a young child on a swing is maintained by periodic pushes
from a parent.
\end{itemize}
If we denote the \textbf{external driving force} by \textit{F(t)} and if we assume as
before that the \textbf{damping force} has the form \textit{-bv}, then the net force on the oscillator
is —bv — kx + F (t) and the equation of motion can be written as shown above.
My name is Sarvesh Mane and my github user-id is 112707983.







