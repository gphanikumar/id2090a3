%\documentclass[12pt,a4paper]{article}

%\usepackage{amsmath,amsfonts,amssymb,graphicx,physics,url}
%\usepackage[left=2cm,right=2cm,top=2cm,bottom=2cm]{geometry}
%\usepackage[colorlinks = true, urlcolor= blue]{hyperref}

%\newcommand{\ft}[1]{\renewcommand\thefootnote{}\footnote{#1}\addtocounter{footnote}{-1}}

%\author{NIRMAL KUMARAN M (ME22B171)}
%\title{\textbf{ASSIGNMENT - 3 LaTex \& Git}}
%\date{February 05, 2023}

%\begin{document}

%\maketitle
\pagebreak

\section{ME22B171}
Name: NIRMAL KUMARAN M \\
Roll No: ME22B171 \\
GitHub ID: NirmalK-M

\subsection{\textbf{Biot–Savart Law}}

\begin{equation}
	%\va{B}(\va{r})= \frac{\mu_{0}}{4\pi} \int_{C}{\frac{{I\va{dl}}\times {\vu{r}}}{|\va{r}|^{2}}}
	\vec{B}(\vec{r})= \frac{\mu_{0}}{4\pi} \int_{C}{\frac{{I\vec{dl}}\times {\hat{r}}}{|\vec{r}|^{2}}} 
\end{equation} \\
\(\vec{B}\) = magnetic field \\ 
\(\mu_{0}\) = permeability of free space \\ 
I = current passing through wire  \\ 
C = line integral along path of wire \\
\(\vec{dl}\) = is in the same direction as the current I (assumed positive)\\ 
\(\vec{r}\) = vector pointing from element dl to position in space 

\subsection{Theory}
\quad The Biot-Savart law expresses the partial contribution $\vec{dB}$ from a small segment of conductor to the total B field of a current in the conductor. $\vec{B}$ points in the direction of superposition (vector sum) of all $\vec{dl}$ $\times$ $\vec{r}$. 

The Biot-Savart Law relates magnetic fields to the steady currents which are their sources, in a similar manner to Coulomb's law which relates electric fields to the point charges which are their sources. Finding the magnetic field involves the vector product, and is inherently a calculus problem when the distance from the current to the field point is continuously changing. It is named after Jean-Baptiste Biot and Felix Savart, who discovered this relationship in 1820. 

\subsection{Condition}
\quad When magnetostatics does not apply, the Biot–Savart law should be replaced by Jefimenko's equations. The law is valid in the magnetostatic approximation, and consistent with both Ampere's circuital law and Gauss's law for magnetism.
\footnotetext{\textbf{References for Biot-Savart Law}}
\footnotetext[1]{Griffiths, David J., "Magnetostatics", Introduction to Electrodynamics (3rd edition), Prentice Hall, September 1999, ISBN 0-13-805326-X, pp. 215-216}

\footnotetext[2]{https://en.wikipedia.org/wiki/Biot-Savart\_law} 

\footnotetext[3]{http://hyperphysics.phy-astr.gsu.edu/hbase/magnetic/Biosav.html}

\pagebreak
%\end{document}
