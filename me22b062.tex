

\begin{center}

\section*{ME22B062}
Name : Krishna
\\
GitHUb ID : Krishna13500
\\
\Large{ID2090 ASSIGNMENT 3}

\Large{Young's Modulus}
\end{center}

\section{Definition}
Young's modulus\footnote{https://en.wikipedia.org/wiki/Young} E, or the modulus of elasticity in tension or compression (i.e., negative tension), is a mechanical property that measures the tensile or compressive stiffness of a solid material when the force is applied lengthwise. It quantifies the relationship between tensile/compressive stress 
$\sigma$ (force per unit area) and axial strain 
$\varepsilon$  (proportional deformation) in the linear elastic region of a material
\section{How to Calculate}
Young's modulus E, can be calculated by dividing the tensile stress,$\sigma$($\varepsilon$), by the engineering extensional strain,$\varepsilon$, in the elastic (initial, linear) portion of the physical stress–strain curve:
\begin{equation}
E = \frac{\sigma(\varepsilon)}{\varepsilon} = \frac{F/A}{\Delta L/L} = \frac{FL}{A\Delta L}
\end{equation}
where
\begin{itemize}
    \item E is Young's modulus (modulus of elasticity)
    \item F is force exerted on object under tension
    \item A is the cross-sectional area
    \item $\Delta$L is the amount by which the length of the object changes ($\Delta$L is positive if the material is stretched, and negative when the material is compressed
    \item L is the original length of object

\end{itemize}
\section{Some Approximate E Values}
Aluminium - 68GPa \\
Brass - 106GPa \\
Bronze - 112GPa \\ 
Carbon Nitride - 822GPa \\
Copper - 110GPa \\
Diamond - 1110GPa
