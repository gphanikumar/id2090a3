\section{Friedmann Equation}
     The Friedmann equations are a set of equations in physical cosmology that govern the expansion of space in homogeneous and isotropic models of the universe within the context of general relativity. They were first derived by Alexander Friedmann in 1922 from Einstein's field equations of gravitation for the Friedmann–Lemaître–Robertson–Walker metric and a perfect fluid with a given mass density d and pressure p.The equations for negative spatial curvature were given by Friedmann in 1924.
     
     The Friedmann equations start with the simplifying assumption that the universe is spatially homogeneous and isotropic, that is, the cosmological principle; empirically, this is justified on scales larger than the order of 100 Mpc.
\subsection{Equation}

\begin{equation}
    H^2 = \frac{8\pi G}{3} \rho - \frac{k}{a^2} + \frac{\Lambda}{3}
\end{equation}


H is Hubble Parameter
 
G is the gravitational constant

$
\rho is the density of matter in the universe
$
 
k is the curvature constant (with k = -1, 0, 1 corresponding to an open, flat, 
   or closed universe, respectively)

 
a is the scale factor

$
\lambda is the cosmological constant
$

\footnote{https://en.wikipedia.org/wiki/Friedmannequations}

Name:Parag Jangle

Github id: ParagJ12
