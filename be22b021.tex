\section*{BE22B021}
\begin{itemize}
    \item Name: Atharv Shete 
    \item Roll Number: BE22B021
    \item GitHub ID : AtharvShete
\end{itemize}

In general relativity, the effective gravitational potential energy of an object of mass m rotating around a massive central body M is given by
\begin{equation}
    U_f (r) = - \frac{GMm}{r} + \frac{L^2}{2mr^2} - \frac{GML^2}{mc^2r^3}
\end{equation}

A conservative total force can then be obtained as
\begin{equation}
    F_f (r) = - \frac{GMm}{r^2} + \frac{L^2}{mr^3} - \frac{3GML^2}{mc^2r^4}
\end{equation}
where L is the angular momentum. The first term represents the Newton's force of gravity, which is described by the inverse-square law. The second term represents the centrifugal force in the circular motion. The third term represents the relativistic effect.
\footnote{
Weinberg, Steven (1972). Gravitation and Cosmology: Principles and Applications of the General Theory of Relativity. John Wiley \\ \\
Cheng, Ta-Pei (2005). Relativity, Gravitation and Cosmology: a Basic Introduction. Oxford and New York: Oxford University Press.
}
