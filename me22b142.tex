\section{ME22B142}
\subsection{Sum of Sines}
\begin{equation}
    \sum_{n=1}^{n} sin(\alpha+(i-1)\beta) = \frac{sin(\alpha+\frac{(n-1)\beta}{2})sin(\frac{n\beta}{2})}{sin(\frac{\beta}{2})}
\end{equation}
This expression is derived from a much simpler and general summation of 2 sines.
\begin{equation}
    sinA + sinB = 2sin(\frac{A+B}{2})cos(\frac{A-B}{2})
\end{equation}
These formulae are used almost all the time when finding solutions of triangles, a major part of the JEE Advanced examination. A very beautiful relation indeed because such a big trigonometric sum can be written in such a short format. I looked up the equation when studying for JEE in NCERT. \footnote{\href{https://ncert.nic.in/textbook.php?kemh1=3-16}{NCERT Book link on NCERT Website}}

Name- Kanupriya Virdi | Roll number- ME22B142 | GitHub user ID- kanupriyavirdi