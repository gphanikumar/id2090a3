\section*{ME22B165}
 In this subsection we express \textit{r} as a function of $\theta$, for a gravitational potential. Let’s assume that we’re dealing with the earth and the sun, with masses \textit{M} and \textit{m} respectively. The gravitational potential energy of the earth–sun
system is
\[V(r)=\frac{-\alpha}{r}\] where $\alpha=GMm$.
\\In the present treatment, we’ll consider the sun to be bolted down at the origin of our coordinate system. For such a system, the \textbf{generalised orbit equation} is given by
\[r(\theta)=\frac{L^2}{m\alpha(1+\epsilon cos\theta)}\]
where  \[\epsilon \equiv \sqrt{{1+\frac{2EL^2}{m\alpha^2}}}\] is the \textit{eccentricity} of the particle’s motion.\footnote{David Morin, \textit{Introduction to Classical Mechanics} (Cambridge: Cambridge University Press, 2008), p.288.}
\\As is clear from the equation, the path depends on the total energy \textit{E} and angular momentum \textit{L} of the satellite. For different eccentricities, different conic sections are described. The following are the cases for $\epsilon$.
\begin{itemize}
\item \textbf{Circle} ($\epsilon$ = 0)
\item \textbf{Ellipse} (0 \textless $\epsilon$ \textless 1)
\item \textbf{Parabola} ($\epsilon$ = 1)
\item \textbf{Hyperbola} ($\epsilon$ \textgreater 1)
\end{itemize}
Name: \textbf{Narayani Adane}
\\GitHub ID: \textbf{NarayaniVA}
