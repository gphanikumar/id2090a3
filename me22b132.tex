\title{An equation that holds a universal truth}
\author{Hridyansh Aggarwal}
\date{07 February 2023}

\begin{document}

\maketitle

\section{Introduction}

This is a very simple quadratic equation that even a child can solve using the quadratic formula, but still pulls out a result so majestic, that even nature gives up to the beauty of it. I remember looking at this equation all the time in school books, but the significance of this equation was brought forward to me by Numberphile. \footnote{Numberphile on YouTube- https://www.youtube.com/@numberphile}

\section{ME22B132}

\begin{equation}
    x^2-x-1=0
\end{equation}

On solving this equation using the quadratic formula, we get two roots. One of them being positive and the other one being negative. The positive root is commonly known as the "Golden Ratio".
\begin{equation}
    \phi = \frac{1+\sqrt{5}}{2}
\end{equation}
It is natural to think of the question that what even is so special about this equation that it has been named as "Golden".
There are a lot of reasons or stories to explain this. Let us talk about some of them.
\begin{enumerate}
    \item Irrationality:- This number is the most irrational number in existence. A very standard way of actually proving this is neatly explained by Ben Sparks on Numberphile. \footnote{Numberphile Video- https://www.youtube.com/watch?v=sj8Sg8qnjOg}
    \item Appearance in the Fibonacci Sequence:- If we take the ratio of consecutive numbers appearing in the Fibonacci Sequence (0, 1, 1, 2, 3, 5......), the ratio tends to approach the golden ratio as the numbers in the sequence approach to infinity. A very interesting result indeed considering the fact that both these entities are highly distinguished from each other. Watch this video of Derek Muller to get an idea of what I am trying to explain. \footnote{Veritasium Video- https://www.youtube.com/watch?v=48sCx-wBs34 | Jump to 12:28 to get to the relevant content}
    \item Nature loves this number as well:- Watch the video in pointer 1 to know the same. (Please watch the whole video and get to know some really cool things.)
\end{enumerate}
Name- Hridyansh Aggarwal | Roll number- ME22B132 | GitHub user ID- HridyanshME22B132
\end{document}