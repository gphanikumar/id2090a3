\section{Introduction of Gauss's law}
In physics and electromagnetism, Gauss's law, also known as Gauss's flux theorem, is a law relating the distribution of electric charge to the resulting electric field. In its integral form, it states that the flux of the electric field out of an arbitrary closed surface is proportional to the electric charge enclosed by the surface, irrespective of how that charge is distributed. Even though the law alone is insufficient to determine the electric field across a surface enclosing any charge distribution, this may be possible in cases where symmetry mandates uniformity of the field.

The law was first formulated by Joseph-Louis Lagrange in 1773, followed by Carl Friedrich Gauss in 1835, both in the context of the attraction of ellipsoids. It is one of Maxwell's four equations, which forms the basis of classical electrodynamics\footnote{This information is referred from wikipedia link: \url{https://en.wikipedia.org/wiki/Gauss\%27s\_law}}.
\section{Equation of Gauss's law}

\[ \oiint\limits_S E.dA=Q/{\varepsilon} \]
where,\\
\indent S=\textit{gaussian surface}\\
\indent E=\textit{electric field over the surface}\\
\indent Q=\textit{total charge enclosed by the surface}\\
\indent $\varepsilon$=\textit{permittivity of the medium in which the charges are present}\\ \\
\indent \textit{Git-id:GB2k4-me22b125}
