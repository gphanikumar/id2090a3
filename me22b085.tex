\section{Introduction}
In mathematics, a quartic equation is one which can be expressed as a quartic function equaling zero. The general form of a quartic equation is
Graph of a polynomial function of degree 4, with its 4 roots and 3 critical points.
\begin{equation}
    ax^{4}+bx^{3}+cx^{2}+dx+e=0\
\end{equation}
where 
\begin{equation}
    a \neq 0 
\end{equation}
The quartic is the highest order polynomial equation that can be solved by radicals in the general case (i.e., one in which the coefficients can take any value).
\subsection{Solving a quartic equation, special cases}
Consider a quartic equation expressed in the form 
\begin{equation}
    a_{0}x^{4}+a_{1}x^{3}+a_{2}x^{2}+a_{3}x+a_{4}=0
\end{equation}

There exists a general formula for finding the roots to quartic equations, provided the coefficient of the leading term is non-zero. However, since the general method is quite complex and susceptible to errors in execution, it is better to apply one of the special cases listed below if possible.
\subsection{Degenerate case}
If the constant term 
\begin{equation}
    a^{4} = 0
\end{equation}
then one of the roots is 
\begin{equation}
    x = 0
\end{equation}
and the other roots can be found by dividing by x, and solving the resulting cubic equation,
\begin{equation}
     a_{0}x^{3}+a_{1}x^{2}+a_{2}x+a_{3}=0
\end{equation}
\subsection{Biquadratic equations}
A quartic equation where a3 and a1 are equal to 0 takes the form
\begin{equation}
    a_{0}x^{4}+a_{2}x^{2}+a_{4}=0
\end{equation}
and thus is a biquadratic equation, which is easy to solve: let 
\begin{equation}
    z=x^{2}
\end{equation}
so our equation turns to
\begin{equation}
    a_{0}z^{2}+a_{2}z+a_{4}=0
\end{equation}
which is a simple quadratic equation, whose solutions are easily found using the quadratic formula:
\begin{equation}
    z={\frac {-a_{2}\pm {\sqrt {a_{2}^{2}-4a_{0}a_{4}}}}{2a_{0}}}
\end{equation}
When we've solved it (i.e. found these two z values), we can extract x from them
\begin{equation}
    {x_{1}=+{\sqrt {z_{+}}}}
\end{equation}
\begin{equation}
    {x_{2}=-{\sqrt {z_{+}}}}
\end{equation}
\begin{equation}
    {x_{3}=+{\sqrt {z_{-}}}}
\end{equation}
\begin{equation}
      {x_{4}=-{\sqrt {z_{-}}}}
\end{equation}
If either of the z solutions were negative or complex numbers, then some of the x solutions are complex numbers.
\subsection{Quasi-symmetric equations}
\begin{equation}
    a_{0}x^{4}+a_{1}x^{3}+a_{2}x^{2}+a_{1}mx+a_{0}m^{2}=0
\end{equation}
Steps:
Divide by \begin{equation}
    x^2
\end{equation}
Use variable change\begin{equation}
    z = x + m/x
\end{equation}
If the quartic has a double root, it can be found by taking the polynomial greatest common divisor with its derivative. Then they can be divided out and the resulting quadratic equation solved.
\footnote{\url{https://en.wikipedia.org/wiki/Quartic_equation}}
\end{document}
