\section{me22b058}
\footnote{https://www.britannica.com/science/Plancks-radiation-law}
Planck’s radiation law, a mathematical relationship formulated in 1900 by German physicist Max Planck to explain the spectral-energy distribution of radiation emitted by a black body.
Planck assumed that the sources of radiation are atoms in a state of oscillation and that the vibrational energy of each oscillator may have any of a series of discrete values but never any value between. Planck further assumed that when an oscillator changes from a state of energy E1 to a state of lower energy E2, the discrete amount of energy \(\ E2-E1\), or quantum of radiation, is equal to the product of the frequency of the radiation, symbolized by the Greek letter {\(\nu\)} and a constant \textbf{h}, now called Planck’s constant, that he determined from black body radiation data; i.e.,\(\ E2-E1=h\nu\)\\
\\
Planck’s law for the energy E radiated per unit volume by a cavity of a black body in the wavelength interval \(\lambda\) to (\(\lambda\) + \(\Delta\) \(\lambda\))  can be written in terms of Planck’s constant (h), the speed of light (c), the Boltzmann constant (k), and the absolute temperature (T):\\
\\
\begin{equation}
  \huge{  E = \frac{8\pi hc}{\lambda^5(e^\frac{hc}{KT\lambda} -1)}}
\end{equation}
\\
\textbf{Name: Shankar K}\\
\textbf{Github User-name: Shankar7804}


