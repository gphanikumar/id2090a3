\section{ME22B180}

\subsection{The Equation}
The Karplus equation can be written as: 


\begin {equation}
\label{kar}
J(\phi)= Acos^2(\phi) + Bcos(\phi) + C \footnote{https://doi.org/10.1016/j.carres.2007.02.023}
\end{equation}

where,
\begin{itemize}
\item J is the \textsuperscript{3}J coupling constant
\item \(\phi\) is the dihedral angle
\item A, B and C are empirically derived constants
\end{itemize}

\subsection{The Explanation}

This equation is used in NMR spectroscopy, usually of proteins. NMR, or Nuclear Magnetic Resonance, is an interesting phenomenon which occurs when a small varying magnetic field disturbs nuclei in a strong magnetic field. J-coupling is a form of interaction between nuclear spins. The superscript 3 indicates that it is between H atoms separated by two C atoms. The value of \(\phi\) varies from 0 to \(\pi\) radians. 

\subsection{Name and Github ID}

\begin{enumerate}
    \item Name: Praveen Joseph Thomas
    \item Github ID: me22b180
\end{enumerate}
